\begin{center}
  {\heiti\zihao{3}利用激光进行微位移测量自动化程序系统的研发}\\[2em]
  {\zihao{5}李正恒}\\[0.5em]
  {\zihao{-5}(物理与光电工程学院\, 应用物理学\, 2010级6班\, 20102312772)}\\
\end{center}
\begin{MyAbstract}  %%%%% 中文摘要 %%%%%%%%%%%%%%%%%%%%
利用激光进行微小位移量测量主要有两种方法:一种是几何的方法,比较常见的是光杠杆;另一种是光学的方法,
比如利用光的干涉原理把所测微小位移量的变化转化为干涉条纹的移动。本文主要研究了这两种测量方法所涉及的图像数据
采集与分析方法,前者包括激光斑点位移的自动化测量、数据的在线分析转化和实时存储,后者包括利用计算机程序的
方法对干涉条纹移动数量的动态计数。另外,本文也列举了四个典型的应用程序项目例子,详细地讲解了实际应用的相关测量
仪器设备程序软件系统的设计与组建方法。
 
\end{MyAbstract}
\begin{MyKeywords}  %%%%%% 中文关键词 %%%%%%%%%%%%%%%%%%%%%
  微位移测量;\;亮中心扩展算法;\;多斑点定位;\;干涉条纹移动数量;\;条纹特征识别
\end{MyKeywords}
\vspace{3mm}
\begin{center}
  {\zihao{3}\textbf{Research and Development of Automatic Program System for the Measurement of Micro-displacement \mbox{Using Laser Beam} } }\\[2em]  
  {\zihao{5}Li Zhengheng }\\[0.5em]
  {\zihao{-5}(School of Physics and Optoelectonic Engineering, Applied Physics,
    Class 6 Grade 2010, 20102312772) }\\
\end{center}
\begin{MyAbstract}[en]  %%%%%%%%%%%%%%% 英文摘要 %%%%%%%%%%%%%%%%%%%%%
There are two main ways for the measurement of micro-diplacement using laser beam: one is the 
geometric approach such as optical lever which is very common, the other is an optical method 
such as using light interference principle to transform the changes of tiny displacement into the 
move of interference fringes. This paper mainly studies the image data acquisition and analysis 
methods involved in these two methods, the former including automaticlly measuring the displacement
of laser beam spot, on-line data analysis and conversion, real-time data storage, and the later 
including dynamically counting the moved amount of interference fringes with the computer program
method. In addition, this paper also cites four examples of typical application  projects 
explaining in detail the design and construction method of practical software systems for relevant 
measuring equipment.

\end{MyAbstract}
\begin{MyKeywords}[en]  %%%%%%%%%%%%%% 英文关键词 %%%%%%%%%%%%%%%%%%
measurement of micro-displacement; bright center expansion algorithm; muti-spot positioning; 
moved amount of interference fringes; fringes feature recognition
  
\end{MyKeywords}


